\documentclass{article}
\usepackage{graphicx} % Required for inserting images
\usepackage{amsmath}
\title{An Overview of Finding of Maxima and Minima Of Functions}
\author{Prepared by Vishnu teja Vasam}
\date{\today}
\begin{document}
\maketitle
\tableofcontents\newpage
\section{introduction}

Understanding how to find the maxima and minima of functions is a key concept
in calculus. It helps in identifying the highest or lowest points on a graph, which
can have various applications in \underline{physics, economics, and optimization problems.}\\
\newpage
\section{Mathematical formulation}

To find the \textbf{maxima} and \textbf{minima} of a function f(x), we start by finding it's first derivative:
\begin{center}
    \begin{equation}
        f'(x) = \text{Derivative of } f(x).
    \end{equation}
\end{center}
    Next, we find the critical points by solving:
\begin{center}
    \begin{equation}
        f'(x) = 0.
    \end{equation}
\end{center}

    The second derivative test is used to determine whether a critical point is a maximum or minimum:

\begin{center}
    \begin{equation}
        \text{if }f''(x) > 0,\text{ then it is a minimum. If }f''(x)<o, \text{ then it is a maximum.}
    \end{equation}
\end{center}
\newpage
\section{ Example: Finding the Maxima and Minima of a Quadratic Function}
Consider the function $f(x) = -x^2 + 4x - 3$. We first calculate the derivative and find the critical points.

The \underline{first derivative} is: 
\[f'(x) = -2x + 4.\]

Solving for $f'(x) = 0$:
\[-2x + 4 = 0 \Rightarrow x = 2.\]

The second derivative is:
\[f''(x) = -2.\]

Since $f''(2) < 0$, $x = 2$ is a maximum point.
\begin{figure}[h]
\centering
\includegraphics[width=0.7\textwidth]
{maxima-minima.png}
\caption{Graph showing the maxima of the function f(x) = $f(x) = -x^2 + 4x - 3$.}
\label{fig:graph}
\end{figure}\\
\begin{table}[h]
    \centering
\begin{tabular}{|c|c|}
 \hline
 \textbf{X} & \textbf{f(x)} \\
 \hline
 1 & 0 \\
 2 & 1 \\
 3 & 0 \\
 \hline
\end{tabular}
\caption{Values of f(x) at different points} \label{tab:my_label}
\end{table}
\begin{center}
    As we can see in Figure 1 and Table 1, the maximum value occurs at x = 2.
\end{center}

\end{document}
